\documentclass{article}

\setlength{\parindent}{0pt}
\usepackage[margin=1in]{geometry}

\begin{document}
\section{Serial Communication}
\subsection*{Linux Overview}
In linux, serial communication sits on top of terminal communication. In the olden days, people needed terminals more than any other peripheral, so they decided to write the library and API for it first. Much of this API relied on serial communication, so they just used it or other serial communication.

Serial devices can be found with the ``dmesg'' command. They generally have two interfaces (for legacy reasons): The ttyS.. and the cua... markers. These are under the /dev/ directory for Linux and are essentially interchangeable (but use tty for standards sake).

Communicating to these devices is actually a little difficult... A couple of libraries exist: (v7, 4BSD, XENIX ioctl-based), termio (old), and termios. Additionally, there is a nice C++ API called LibSerial which has made simple classes for serial communication use. If you go the old way, know that the Universal file IO model still holds: you use open, close, read and write to manipulate the data transfer, you just use a lot of other stuff to initialize and manage it.

You can also configure default serial connection qualities using the /etc/ttytab config file. Setting Getty field to ``'' lets the machine know its not a terminal and the Init field can execute a command.

For our purposes, just use libserial.
\subsection*{LibSerial}
Simply create a SerialStream object and set its characteristics.

Note that you must run the program with ``sudo'' to enable access to serial port itself. If you do not, the stream will not open and there is not a good error reporting segment of the library.

Use the read() and write() methods for the SerialStream Object to receive and send arbitrary structures like we want. I still don't know how to get blocking reads...

!!You need to run the executable with ``sudo'' to get the permissions to access the serial port on the computer.

\subsection*{Hookin' up the Circuitry}
As we remembered... The enable pin on the motor controller needs to be run low to enable the motor... So, be sure to do that. Also, you need to set DDRx to all ones to get the microcontroller to output instead of letting it hang. If you don't, it doesn't have the juice to run anything high or low... So you get odd results or apparently dead things. Also, turns out just bridging the grounds works wonderfully, I got ground connectivity across the entire system from stealing a simple sensor ground wire. I would rather use the motor controller's GND test port for this.

\subsection*{Interrupts}
You need to turn on the global interrupt flag (sei()), enable the individual interrupt, and register your functions with each interrupt you intend to capture. All interrupts have flags and vectors: the interrupt flag always gets set and held incase the interrupt is ever requested, and the vector points to a point in memory which contains a reference to the interrupt handler. The compiler handles setting your function there (just use ISR(vect\_name).
Interrupts automatically disable other interrupts from occurring by turning off the global interrupt flag.

\subsection*{Debugging}
To start up a text terminal with the serial port, simply type ``sudo screen /dev/ttyUSB0 9600'', you will get a copy of the serial communications occuring.

\end{document}
